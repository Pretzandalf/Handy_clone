\documentclass[border=0cm]{article}
 
\usepackage{pgfplots}
\usepackage{tikz-3dplot}
\usepackage{amsmath}
 
\tdplotsetmaincoords{60}{115}
\pgfplotsset{compat=newest}
 
\begin{document}
\part*{Ball's aerodynamycs}

\begin{tikzpicture}[tdplot_main_coords, scale = 1.5]

% Axes in 3d coordinate system
% \draw[-stealth] (0,0,0) -- (4,0,0) 
%     node[below left] {$x$};
% \draw[-stealth] (0,0,0) -- (0,8,0)
%     node[below right] {$y$};
% \draw[-stealth] (0,0,0) -- (0,0,8)
%     node[above] {$z$};
 
 
% Create points 
\coordinate (Arc_x) at (0, 3, 4);
\coordinate (Arc_y) at (1, 4, 4);

\coordinate (O) at (1, 3, 4);
\coordinate (F_mg) at (1, 3, 1);
\coordinate (F_air_resistance) at (1.75, 0.75, 2.5);
\coordinate (F_Magnus) at (-1.33, 0.5, 3.92);
\coordinate (Vel_lin) at (0, 6, 6);
\coordinate (Vel_ang_begin) at (0, 3.5, 3);
\coordinate (Vel_ang_end) at (2.5, 2.25, 5.5);


% Draw vectors
\draw[fill = red] (O) circle (1pt);
\draw[->, black, thick] (O) -- (F_mg) node[black, right] {$F_{gravity}$};
\draw[->, red, thick] (O) -- (F_air_resistance) node[red, below] {$F_{resistance}$};
\draw[->, black!30!green, thick] (O) -- (Vel_lin) node[black!30!green, right] {$\upsilon$, linear velocity};
\draw[->, white!30!blue, dashed] (Vel_ang_begin) -- (Vel_ang_end) node[white!30!blue, left] {$\omega$, angular velocity};
\draw[->, cyan, thick] (O) -- (F_Magnus) node[cyan, right] {$F_{Magnus}$};


% Draw ball
\shade[ball color = lightgray,
    opacity = 0.5
] (O) circle (1cm);
 
% draw arcs 
\tdplotsetrotatedcoords{0}{0}{0};
\draw[dashed,
    tdplot_rotated_coords,
    gray
] (O) circle (1);

\tdplotsetrotatedcoords{90}{90}{90};
\draw[dashed,
    tdplot_rotated_coords,
    gray
] (Arc_x) arc (0:180:1);

\tdplotsetrotatedcoords{0}{90}{90};
\draw[dashed,
    tdplot_rotated_coords,
    gray
] (Arc_y) arc (0:180:1);
 
\end{tikzpicture}

\section*{Main forses}
\begin{flalign*}
    &F_{gravity} = mg& \\
    &F_{resistance} = \alpha {\upsilon}^2 - \textit{a force acting opposite to the relative motion of a moving ball.}& \\
    &F_{Magnus} = \beta \begin{bmatrix} \vec \omega \times \vec v \end{bmatrix} - \textit{a force arising from the pressure difference on the walls of the ball.}& 
\end{flalign*}

\section*{Dynamics of the system}
\subsection*{Newton's laws of motion:}

\begin{flalign*}
&
m \begin{bmatrix}
\ddot x \\
\ddot y \\
\ddot z \\
\end{bmatrix}
= - \alpha \begin{bmatrix}
{\dot x}^2 \\
{\dot y}^2 \\
{\dot z}^2 \\
\end{bmatrix} -\beta \begin{bmatrix}
\vec \omega \times \vec v \\
\end{bmatrix} - m \begin{bmatrix}
0 \\
0 \\
g \\
\end{bmatrix}& \\
&
\begin{bmatrix}
\vec \omega \times \vec v \\
\end{bmatrix} 
= \det \begin{bmatrix}
\vec i & \vec j & \vec k \\
\omega_x & \omega_y &\omega_z \\
\dot x & \dot y & \dot z \\
\end{bmatrix} = \begin{bmatrix}
0 \cdot \dot x - \omega_z \dot y +\omega_y \dot z \\
\omega_z \dot x + 0 \cdot \dot y  - \omega_x \dot z \\
- \omega_y \dot x + \omega_x \dot y + 0\cdot \dot z \\
\end{bmatrix} & 
\end{flalign*}

\begin{flalign*}
&
m \begin{bmatrix}
\ddot x \\
\ddot y \\
\ddot z \\
\end{bmatrix} 
= - \begin{bmatrix}
\alpha {\dot x}^2 - \beta \omega_z \dot y + \beta \omega_y \dot z \\
\beta \omega_z \dot x + \alpha {\dot y}^2  - \beta \omega_x \dot z \\
- \beta \omega_y \dot x + \beta \omega_x \dot y + \alpha {\dot z}^2 \\
\end{bmatrix} - m \begin{bmatrix}
0 \\
0 \\
g \\
\end{bmatrix}
&
\end{flalign*}



 
\end{document}